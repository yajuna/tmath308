%  Math 309, Spring 2012
%  Math 309 Autumn 2012
%  Worksheet for reviewing linear algebra at start of quarter
%  Latex document

\documentclass[12pt]{article}
\usepackage{amssymb}
\usepackage{amsmath}
\usepackage{multicol}
\usepackage{graphicx,float,wrapfig}
\newcommand      {\R}         {{\mathbb R}}
\newcommand      {\C}         {{\mathbb C}}
\newcommand      {\vb}        {\mathbf}
\textwidth=7in
\textheight=9.5in
\topmargin=-0.7in
\oddsidemargin= 0.0in
\evensidemargin= 0.0in
\setlength{\parskip}{1ex plus0.2ex minus0.2ex}
\pagestyle{myheadings}
\markright{Tmath 308 \hfill Lab 1: Intro to Octave \hfill 
Summer 2019
\hspace{1cm}}
\begin{document}
(updated:\today) This is our first lab to introduce Matlab/ Octave.

\begin{enumerate}
\item Evaluate the following math expressions in Matlab/Octave
\begin{itemize}\begin{multicols}{3}
\item $\tanh(e)$ \item $\arcsin(-1/2)$ \item $\log_{10}(2)$\item $123456\ mod\ 789$ 
\item $Arg(1+i\sqrt{2})$ \item $|i+1|$
\end{multicols}
\end{itemize}
\item Find the built in functions in Matlab/Octave, and do the following 

\begin{itemize}
\item Compute a Bessel function of the second kind 
\item Test the primality of 482023487
\item Plot a vector field (Note: do not copy and paste examples you find online, but make your own based on the examples.) 
\item Report current date and time

\end{itemize}

\item (Generate vectors) Type in {\tt vec = 1:8}, {\tt vec1 = 1:2:8}, {\tt v2 = 3:-0.5:1} respectively and see what you get. Find a command that gives you the row vector {\tt 5.6 5.4 5.2 ... 3.8 3.6 3.4}.

\item (Generate matrices) Generate some special matrices, find the built in functions in Matlab/Octave.
\begin{itemize}
\item a zero matrix of dimension $2\times 3$; \item a matrix whose elements are all 1's, of dimension $100\times 30$; \item a diagonal matrix of dimension $100\times 100$, whose diagonal elements are 1,2,..., 100; \item find the command for identity matrices, make an identity matrix of size $50000000\times 50000000$ and record the time elapsed for your computer to output the result (mine was simply stuck);\item type in {\tt magic(5)} and see what you get; type in {\tt help magic} and paste what you see here.
\item It would be wrong not to talk about sparse matrices when learning Matlab/Octave\footnote{before sparse matrices were introduced in 1993, Matlab was not quite considered a serious computing language}. Type in {\tt A = sparse(1:4,8:-2:2,[2 3 5 7])} and {\tt AA = full(A)}, copy and paste your results here.Then type {\tt spy(A)} and {\tt spy(AA)}. Include your images here.
\end{itemize}








\end{enumerate}






\end{document}
